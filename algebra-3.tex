\documentclass[a4paper]{article}
\usepackage{cmap}
\usepackage[T2A,T1]{fontenc}
\usepackage[utf8]{inputenc}
\usepackage[russian]{babel}
\usepackage[left=20mm,right=20mm,top=20mm,bottom=20mm,bindingoffset=0cm]{geometry}
\usepackage[pdfborder={0 0 0}]{hyperref}
\usepackage{indentfirst}
\usepackage{pgf}
\usepackage{pgfmath}
\usepackage{tikz}
\usepackage{tikz-cd}
\usepackage{pgffor}
\usepackage{wrapfig}
\usepackage{multicol}
\usepackage{nicefrac}

\usetikzlibrary{matrix,arrows,shapes.geometric,babel,angles}

\makeatletter
\tikzset{
  edge node/.code={%
      \expandafter\def\expandafter\tikz@tonodes\expandafter{\tikz@tonodes #1}}}
\makeatother
\tikzset{
  Subset/.style={
    draw=none,
    every to/.append style={
      edge node={node [sloped, allow upside down, auto=false]{$\subset$}}}
  }
}

\usepackage{amssymb,amsmath,amsthm,amsfonts,amscd}

\newcommand{\R}{\ensuremath{\mathbb{R}}}
\newcommand{\Q}{\ensuremath{\mathbb{Q}}}
\newcommand{\Z}{\ensuremath{\mathbb{Z}}}
\newcommand{\F}{\ensuremath{\mathbb{F}}}

\newcommand{\eps}{\varepsilon}

\let\temp\phi
\let\phi\varphi
\let\varphi\temp

\newcommand{\id}{\mathrm{id}}

\DeclareMathOperator{\chr}{char}

\DeclareMathOperator{\im}{Im}
\DeclareMathOperator{\Aut}{Aut}
\DeclareMathOperator{\Id}{Id}
\DeclareMathOperator{\ord}{ord}
\DeclareMathOperator{\dsep}{d_{sep}}
\DeclareMathOperator{\di}{d_{i}}

\newtheorem{theorem}{Теорема}
\numberwithin{theorem}{section}

\newtheorem{lemma}{Лемма}
\numberwithin{lemma}{section}

\newtheorem*{theorem*}{Теорема}
\newtheorem*{lemma*}{Лемма}

\newtheorem{proposition}{Предложение}
\numberwithin{proposition}{section}

\newtheorem{corollary}{Следствие}
\numberwithin{corollary}{section}

\begin{document}

\section{Расширения полей}

\subsection*{Характеристика поля}
Существует единственный гомоморфизм $\phi: \Z \to K$, $\phi(1) = 1_K$.
Возможны два случая:
\begin{itemize}
    \item $\ker \phi = 0$, тогда $\phi$ можно продолжить до гомоморфизма из $\Q$ в $K$,
    $\phi(\frac{n}{m}) = \frac{n_K}{m_K}$,
    тогда $\Q \subseteq K$ и $\chr K = 0$.
    \item
    \begin{minipage}{0.9\linewidth}
        $\ker \phi \ne 0$, тогда $\phi$ пропускается через $\frac{\Z}{\ker \phi}$, где идеал $\ker \phi$ порождается одним элементом~$p > 0$, и $p$ простое.
        Тогда $\F_p \subseteq K$ и $\chr K = p$.
    \end{minipage}
    \begin{minipage}{0.15\linewidth}
    \begin{tikzcd}[row sep=scriptsize,column sep=tiny]
        \Z \arrow[r, "\phi" above] \arrow[d, ->>] & K \\
        \nicefrac{\Z}{\ker \phi} \arrow[ur, hook]
    \end{tikzcd}
    \end{minipage}
\end{itemize}

\begin{theorem}
Если $K$ --- конечное поле, то $|K| = p^n$ для простого числа $p$ и $0 < n \in \Z$.
\end{theorem}
\begin{proof}
$\chr K = p > 0$, следовательно, $\F_p \subseteq K$ и $K$ является векторным пространством над $\F_p$. Если $\dim_{\F_p}K = n$, то $|K| = p^n$.
\end{proof}

\subsection*{Примеры полей}

Пусть $A$ --- коммутативное кольцо с единицей и $I \subset A$ --- идеал.
\begin{proposition}
$\nicefrac{A}{I}$ --- поле тогда и только тогда, когда $I$ --- максимальный.
\end{proposition}
\begin{proof}
Пусть $I$ максимальный, рассмотрим $\phi: A \to \nicefrac{A}{I}$.
Для $x \in \nicefrac{A}{I}$ прообраз $\phi^{-1}((x))$ является идеалом, следовательно
\[\phi^{-1}((x)) \supsetneq I
\Rightarrow \phi^{-1}((x)) = A
\Rightarrow (x) = \nicefrac{A}{I}
\Rightarrow \exists y : \: xy = 1_{\nicefrac{A}{I}}\]

В обратную сторону: пусть $I \subsetneq I'$. $\phi(I')$ --- идеал в
поле $\nicefrac{A}{I}$, следовательно $\phi(I') = \nicefrac{A}{I} \Rightarrow I' = A$.
\end{proof}

\begin{minipage}{0.7\linewidth}
Частный случай: кольцо $A = K[x]$, $f \in K[x]$ --- неприводимый многочлен, тогда $I = (f)$ --- максимальный идеал,
а $\nicefrac{K[x]}{(f)}$ --- поле.
\end{minipage}
\begin{minipage}{0.15\linewidth}
\begin{tikzcd}[row sep=scriptsize,column sep=scriptsize]
    K[x] \arrow[r, "\phi" above] & \nicefrac{K[x]}{(f)} \\
    K \arrow[Subset]{u}{} \arrow[ur, hook]
\end{tikzcd}
\end{minipage}

\textbf{Замечание.}
Существует $\alpha \in \nicefrac{K[x]}{(f)}$, такая что $f(\alpha) = 0$. Если $f(x) = a_d x^d + \ldots + a_1 x + a_0 $ и $\alpha = \phi(x)$,
то $f(\alpha) = f(\phi(x)) = \phi(f(x)) = 0$.

Любой многочлен $g(x) \in K[x]$ можно единственным образом записать как $g(x) = f(x)q(x) + r(x)$, где $\deg r(x) < d$, поэтому
$\{1_K, \phi(x), (\phi(x))^2, \ldots, (\phi(x))^{d-1}\}$ --- базис в
\nicefrac{K[x]}{(f)}.

\subsection*{Расширения полей}
Поле $L$ называется \textbf{расширением} $K$, если $K \subset L$.
Размерность $\dim_K L$ называется \textbf{степенью расширения} --- $[L : K]$.
Если степень $[L : K] < \infty$, то расширение называется \textbf{конечным}.

\begin{proposition} В цепочке расширения $K \subset L \subset F$
\[[F:K] = [F:L] \cdot [L : K]\]
\end{proposition}
\begin{proof}
Пусть $v_1, v_2, \ldots, v_\alpha$ --- базис в $L$ над $K$ и
$w_1, w_2, \ldots, w_\alpha$ --- базис в $F$ над $L$.
Тогда $\{v_i w_j\}$ --- базис в $F$ на $K$.
Действительно, пусть $\alpha \in F$. Тогда
\[\alpha = \sum_j b_j w_j = \sum_j \sum_i a_{ij}v_i w_j \]

Чтобы доказать линейную независимость, рассмотрим
линейную комбинацию $\sum_{i,j} a_{ij}v_i w_j = 0$, где $a_{ij} \in K$.
Переупорядочивая, получим \[\sum_i\left(\sum_j a_{ij} w_j\right)v_i = 0,\]
следовательно \[\forall i \quad \sum_j a_{ij} w_j = 0\]
Отсюда все $a_{ij} = 0$.
\end{proof}

\subsection*{Алгебраические расширения}
Пусть $K \subset L$ --- расширение поле, тогда $\alpha \in L$
называется \textbf{алгебраическим} над $K$, если существует ненулевой
$f \in K[x]$, такой что $f(\alpha) = 0$.

Рассмотрим гомоморфизм $ev_\alpha: K[x] \to L$, $ev_\alpha(f(x)) = f(\alpha)$. Тогда $\alpha$ алгебраический тогда и только тогда,
когда $\ker \alpha \ne 0$.

Если $\alpha$ --- алгебраический над $K$, то $\ker ev_\alpha = (f)$,
где $f$ --- неприводимый и $f(\alpha) = 0$. Такой $f$ называется
\textbf{минимальным многочленом} для $\alpha$. Его степень $\deg f$ обозначается $\deg_K f$.

Для расширения $K \subset L$ и
элементов $\alpha_1, \dots, \alpha_r \in L$
обозначим как $K(\alpha_1, \dots, \alpha_r)$ наименьшее подполе $L$,
содержащее все $\alpha_1, \dots, \alpha_r$.

\begin{proposition}
$\alpha \in L$ алгебраичен над $K$ тогда и только тогда,
когда $[K(\alpha) : K] < \infty$.
\end{proposition}
\begin{proof}
$\Leftarrow$: $K[x] \xrightarrow{ev_\alpha} K(\alpha) \subset L$
$\dim_K K(\alpha) < \infty$, $\dim_K K[x] = \infty$, следовательно
$\ker ev_\alpha \ne 0$ и $\alpha$~---~алгебраический.

$\Rightarrow$: Если $\alpha$ --- алгебраический, то $[K(\alpha) : K] = \deg f$, где $f$ --- минимальный многочлен для $\alpha$.
\end{proof}

Отсюда по индукции получаем, что если $\alpha_1, \dots, \alpha_r$ алгебраические, то $[K(\alpha_1, \dots, \alpha_r) : K] < \infty$.

\begin{theorem}
Пусть $K \subset L$ --- расширение.
Рассмотрим $L_{alg} = \{\alpha \in L \mid  \text{$\alpha$ алгебраично над $K$}\}$.
Тогда $L'$ --- подполе $L$.
\end{theorem}
\begin{proof}
Пусть $\alpha, \beta \in L_{alg}$.
Поскольку $[K(\alpha, \beta) : K] < \infty$, каждый элемент $\gamma \in K(\alpha, \beta)$ алгебраичен над $K$.
\end{proof}

\section{Невозможность некоторых построений}
При построених циркулем и линейкой нам разрешены три операции:
\begin{enumerate}
    \item Провести прямую через ранее построенные точки
    \item Провести окружность с центром в ранее построенной точке через другую точку.
    \item Построить точку пересечения
\end{enumerate}

\begin{theorem}
Дан отрезок длины 1. С помощью циркуля и линейки можно построить отрезок длины $a \in \R$, где $a$ --- квадратичная иррациональность.
\end{theorem}
\begin{proof}
Докажем, что если точку с координатами $(a, b)$ можно построить,
то $a$ и $b$ --- квадратичные иррациональности.

%TODO
(картинка с кругом)

Если $(x, y)$ является пересечением прямой через и окружности,
то $(x, y)$ является решением системы
\[
    \begin{cases}
        (x - a)^2 + (y - b)^2 = r^2 \\
        cx+dy = e
    \end{cases},
\]
где $a, b, r, c,d,e$ --- квадратичные иррациональности.

%TODO
Если $\alpha = x + iy$ является пересечением двух окружностей,
то $(x, y)$ является решением системы
\[
    \begin{cases}
        (x - a)^2 + (y - b)^2 = r^2 \\
        (x - c)^2 + (y - y_3)^2 = x_4^2 + y_4^2
    \end{cases},
\]
где $a, b, r, c,d, R$ --- квадратичные иррациональности.
\[
    \begin{cases}
        (x - a)^2 + (y - b)^2 = r^2 \\
        2(a - c)x + 2(b - d)y = R^2 - r^2 + a^2 + b^2 -c^2-d^2
    \end{cases}
\]
и решение --- квадратичная иррациональность.
\end{proof}

\subsubsection*{Напоминание}
Построим, например, $\sqrt{a}$, если дан отрезок длины $a$.
%TODO

\subsubsection*{Наблюдение}
Для квадратичной иррациональности $a \in \R$
степень $\deg_\Q a = 2^n$.

\subsubsection*{Пример}
Пусть $a = \dfrac{\sqrt{\sqrt{2}+\sqrt{3}}}{\sqrt{7}}$.

%TODO
\[\Q \subset \Q(\sqrt{2}) \subset \Q(\sqrt{2},\sqrt{3})
\subset \Q(\sqrt{2},\sqrt{3},\sqrt{7})
\subset \Q(\sqrt{2},\sqrt{3},\sqrt{7}, \sqrt{\sqrt{2}+\sqrt{3}})\]

Поскольку степень $[K_i : K_{i - 1}]$ равна 1 или 2, то $[K_4 : \Q]$ --- степень 2.
Поскольку $\Q \subset \Q(a) \subset K_4$,
\[[\Q(a) : \Q] \cdot [K_4 : \Q(a)] = [K_4 : \Q] = 2^k
\Rightarrow [\Q(a) : \Q] = 2^{k'}\]

\subsubsection*{Следствия}
Нельзя построить $\sqrt[3]{2}$: многочлен $x^3 - 2$ неприводим по критерию Эйзенштейна и $\deg_\Q \sqrt[3]{2} = 3$.

Нельзя построить семиугольник:

\begin{minipage}{0.35\linewidth}
\centering
\begin{tikzpicture}
\node[draw=none,minimum size=4cm,regular polygon,regular polygon sides=7,draw,shape border rotate=12.8571429] (a) {};
\foreach \x in {1,2,...,7}
  \fill (a.corner \x) circle[radius=2pt];
\draw (0,0) circle (2cm);
\draw (0,0) node[circle,fill,inner sep=1pt](O){};
\draw (O) edge[thick] (a.corner 7);
\draw (O) edge[thick] (a.corner 1);
\draw (a.corner 7) coordinate (A) -- (0,0) coordinate (O) -- (a.corner 1) coordinate (C) pic[draw,->,angle eccentricity=1.8,"$\frac{2\pi}{7}$"{font=\footnotesize}] {angle = A--O--C};
\end{tikzpicture}
\end{minipage}
\begin{minipage}{0.6\linewidth}
Достаточно доказать, что \[\deg_\Q \cos \frac{2\pi}{7} \ne 2^n \]
Пусть $\xi = \exp(\frac{2\pi i}{7}) = \cos(\frac{2\pi}{7}) + i \sin(\frac{2\pi}{7}$, $\xi^7=1$.

Тогда $\cos(\frac{2\pi}{7}) = \frac{1}{2}(\xi + \xi^{-1})$. Многочлен $1+x+x^6$ --- неприводимый, отсюда $\deg_\Q \xi = 6$.

Поскольку $\xi^2 - 2 \cos (\frac{2\pi}{7})\xi + 1 = 0$, степень
$[\Q(\xi) : \Q(\cos (\frac{2\pi}{7}))]$ равна 2 или 1.
Поэтому $[\Q(\cos (\frac{2\pi}{7})) : \Q] = 3 \ne 2^n$.
\end{minipage}

\subsection*{Алгебраические расширения}
Расширение $K \subset L$ называется \textbf{алгебраическим},
если $L_{alg} = L$.

\begin{proposition}
Если $L$ --- алгебраическое расширение $K$ и $F$ --- алгебраическое расширение $L$, то расширение $F \subset F$ также алгебраическое.
\end{proposition}

\begin{proof}
Пусть $\alpha \in F$. Тогда существует $f = a_0 + a_1 x + \ldots + a_d x^d \in L[x]$, такой что $f(\alpha) = 0$. Все $a_i$ алгебраичны над $K$
%TODO
\[K \subset K(a_0, a_1, \ldots, a_d) \subset K(a_0, a_1, \ldots, a_d, \alpha)\]
\end{proof}

\subsubsection*{Поле разложения}

Пусть $f \in K[x]$. Тогда $L \supset K$ называется \textbf{полем разложения} $f$, если $f$ разлагается на линейные множители в $L[x]$,
то есть $f(x) = a(x - a_1) \cdots (x - \alpha_d)$,
где $\alpha_i \in L$, и $L = K(\alpha_1, \ldots, \alpha_d)$.

\begin{proposition}
Поле разложения существует
\end{proposition}
\begin{proof}
Пусть $g \in K[x]$ --- неприводимый делитель $f$.
Рассмотрим $K_1 = \frac{K[x]}{(g)} \supset K$. В $K_1$ многочлен $g$
имеет корень $\alpha = x + (g)$. Применим такую же конструкцию
ко всем неприводимым множителям $\frac{f(y)}{y - \alpha} \in K_1[y]$
и возьмём $L = K(\alpha_1, \ldots, \alpha_d)$
\end{proof}

\section{Алгебраическое замыкание}
\textbf{Алгебраическим замыканием} поля $K$ называется алгебраически замкнутое поле $\overline{K}$, алгебраичное над $K$.

\textbf{Замечание.} Если $L \supset \overline{K}$ --- алгебраическое расширение, то $L = \overline{K}$.

\begin{theorem}
Алгебраическое замыкание существует
\end{theorem}
\begin{proof}
Пусть $I$ --- множество всех многочленов $f \in K[x]$ положительной степени. Вполне упорядочим $I$ и построим по индукции цепочку
расширений $K \subset K_\alpha \subset K_\beta \subset \ldots$,
где $\alpha \prec \beta$ и $K_\beta$ --- это поле разложения
$f_\beta \in I$ над $\bigcup_{\alpha\prec\beta}K_\alpha$.
Положим $\overline{K} = \bigcup_{\gamma \in I} K_\gamma$ и докажем,
что это алгебраическое замыкание. По построению $\overline{K}$ является алгебраическим расширением $K$.
Пусть $\overline{K} \supset L$ --- алгебраическое расширения.
Тогда для $z \in L$
\[K \subset \overline{K} \subset\overline{K}(z) \subset L\]

Элемент $z$ алгебраичен над $\overline{K}$, следовательно, и над $K$,
то есть существует многочлен $f \ne 0$, такой что $f(z) = 0$.
Но тогда $z \in \overline{K}$ по построению.
\end{proof}

\begin{proposition}
Алгебраическое замыкание единственно с точностью до изоморфизма
\end{proposition}

Пусть $L_1, L_2 \supset K$ --- расширения.
Обозначим морфизмы над $K$

\[Mor_K(L_1, L_2) = \{\phi: L_1 \to L_2 \mid \phi|_K = \id_K \}\]

\[
\begin{tikzcd}
    L_1 \arrow[r, "\phi" above] & L_2 \\
    K \arrow[Subset]{u}{} \arrow[r, "\id"] & K \arrow[Subset]{u}{}
\end{tikzcd}
\]

\subsubsection*{Наблюдения}
Если $\alpha \in L_1$ и $f(\alpha) = 0$ для $f \in K[x]$,
то $f(\phi(\alpha)) = 0$. Отсюда следует,
что если $[ L_1 : K ] < \infty$, то $|Mor_K(L_1,L_2)|<\infty$.

Пусть $L_1 = \frac{K[y]}{(f)}$, где $f$ --- неприводимый.
Тогда морфизмы $Mor_K(L_1, L_2)$ соответствуют корням многочлена $f$ в $L_2$.

\paragraph*{Пример} $|Mor_\Q(\Q(\sqrt[3]{3}, \mathbb{C}))| = 3$.

\begin{theorem}
Пусть $\overline{K} \supset K$ --- алгебраическое замыкание
и $L \supset K$ --- алгебраическое расширение. Тогда существует
вложение $L \supset \overline{K}$, то есть $Mor_K(L, \overline{K}) \ne \emptyset$.
\end{theorem}
\begin{proof}
По лемме Цорна существует максимальный элемент порядка $(L', \phi)$,
где $K \subset L' \subset L$ и $\phi: L' \hookrightarrow \overline{K}$,
такой, что $\phi|_K = \id$.
Покажем, что $L' = L$.
Пусть $\alpha \in L \setminus L'$.
Рассмотрим его минимальный многочлен $f \in L'[x]$
и поле $L'(\alpha) = \frac{L'[x]}{(f)}$. Определим
гомоморфизм $\phi: \frac{L'[x]}{(f)} \to \overline{K}$, $\phi: x \mapsto \alpha \in \overline{K}$, противоречие с максимальностью.
\end{proof}

Следствие --- $|Mor_K(L, \overline{K})| \le [L : K]$. В самом деле,
пусть $L = K(\alpha_1, \ldots, \alpha_m)$. Тогда при добавлении $\alpha_i$ возможно не более чем $\deg_{K(\alpha_1, \ldots, \alpha_{i-1}} \alpha_i$ вложений в $\overline{K}$.

\[
\begin{tikzcd}
    K(\alpha_1, \ldots, \alpha_{i+1}) \arrow[r, hook] & \overline{K} \\
    K(\alpha_1, \ldots, \alpha_i) \arrow[Subset]{u}{} \arrow[ru, hook] &
\end{tikzcd}
\]

\begin{proposition}
Поле разложения единственно.
\end{proposition}
\begin{proof}
Пусть $f \in K[x]$ и $L_1$, $L_2$ --- поля разложения $f$ над $K$.

Существует $\phi$, переводящий корни $f$ в $L_1$ в корни $f$ в $\overline{L_2}$, поскольку это алгебраическое замыкание. Они лежат в $L_2$, следовательно, $\phi(L_1) \subset L_2$.

\[
\begin{tikzcd}
    &  \overline{L_2} \\
    L_1 \arrow[ur, "\exists \phi"] \arrow[r, dash] & L_2 \arrow[Subset]{u}{}
\end{tikzcd}
\]

Пусть $f(x) = a(x - \alpha_1)\cdots(x - \alpha_d) \in L_1[x]$, тогда
$f(x) = a(x - \phi(\alpha_1))\cdots(x - \phi(\alpha_d)) \in L_2[x]$,
то есть все корни $f$ в $L_2$ лежат в $\phi(L_1)$, следовательно,
все корни $f$ в $L_2$ лежат в $\phi(L_1)$ и $\phi(L_1) = L_2$
\end{proof}

\begin{proposition}
Алгебраическое замыкание единственно.
\end{proposition}
\begin{proof}
\begin{minipage}{0.15\linewidth}
\begin{tikzcd}
    \overline{K} \arrow[r, hook, "\exists \phi"] & \overline{K'} \\
    K \arrow[Subset]{u}{} & K \arrow[Subset]{u}{}
\end{tikzcd}
\end{minipage}
\begin{minipage}{0.63\linewidth}
Докажем, что $\phi$ сюръективно. Пусть $\alpha \in \overline{K'}$.
Пусть $f$ --- минимальный многочлен для $f$. В обоих алгебраических
замыканиях есть подполя, являющиеся полями разложения $f$, в частности,
такое подполе есть в $\overline{K}$. В нём есть такой корень $f$,
который переходит в $\alpha$.
\end{minipage}
\end{proof}

\subsection*{Сепарабельность}

Многочлен $f \in K[x]$ называется $\textbf{сепарабельным}$,
если у его нет кратных корней в $\overline{K}$.

\begin{lemma}
$\alpha$ --- кратный корень $f$ тогда и только тогда,
когда $f(\alpha) = f'(\alpha) = 0$
\end{lemma}
\begin{proof}
Если $f(\alpha) = 0$,
то $f(x) = a_1(x - \alpha) + a_2 (x - \alpha)^2 + \ldots$.
Тогда $f'(\alpha) = a_1$
\end{proof}
\begin{lemma}
$f$ сепарабелен тогда и только тогда,
когда $f$ и $f'$ не имеют общих делителей степени больше 0 в $K[x]$.
\end{lemma}
\begin{proof}
Пусть $\gcd(f, f') = 1$. Тогда существуют $a, b \in K[x]$, такие,
что $af + bf' = 1$, и следовательно $f$ и $f'$ взаимно просты и в
$\overline{K}$
\end{proof}

\begin{theorem}
Пусть $\chr K = 0$. Тогда любой неприводимый $f \in K[x]$ сепарабельный.
\end{theorem}
\begin{proof}
Если $f = a_nx^n + \ldots + a_0$, то $f' = n a_nx^{n-1}+\ldots+a_1 \ne 0$, поскольку $\chr K = 0$, и $\gcd(f, f') = 1$.
\end{proof}


\section{Теорема о продолжении гомоморфизма}
Пусть $L \supset K$ --- алгебраическое расширение и $F$ --- алгебраически замкнуто поле, содержащее $K$.
Тогда $\exists \phi: L \to F$, такой, что $\phi|_K = \id|_K$.
Более того, таких вложений не больше, чем $[L : K]$.

\paragraph*{Пример.} Пусть $L = K(\alpha)$,
где $\alpha$ --- алгебраическое.
Тогда $\phi: L \to F$ определяется образом $\alpha$
и $\phi(\alpha)$ --- корень минимального многочлена $\mu_d(x)$.
Отсюда получаем, что различных $\phi$ не больше, чем количество
корней $\mu_d(x) = [K(\alpha) : K]$ (и в точности равно количеству различных корней).

% \paragraph*{Замечание.} Если $L$ --

\subsection*{Нормальные расширения}

Алгебраическое расширение $L \supset K$ называется \textbf{нормальным},
если любо неприводимый многочлен $f \in K[x]$, имеющий корень в $L$,
разлагается в $L$ на линейные множители.

\begin{proposition}
Поле разложения нормально.
\end{proposition}
\begin{proof}
Пусть $L \supset K$ --- поле разложения $f$.
Пусть $g \in K[x]$ неприводимый и $\alpha \in L$ --- его корень.
Пусть $\beta$ --- другой корень $g$, тогда существует
гомоморфизм $\phi: L \to F$, продолжающий
гомоморфизм $\psi: K(\alpha) \to F$, $\psi: \alpha \mapsto \beta$.
Поскольку $\phi(L) \subset L \subset F$, получаем, что $\beta \in L$
\end{proof}

\subsection*{Расширения Галуа}

Расширение называется сепарабельным,
если минимальный многочлен каждого его элемента является сепарабельным.

Расширение $L \supset K$ называется \textbf{расширением Галуа},
если $L$ нормально и сепарабельно.

\textbf{Группой Галуа} расширения $L \supset K$ называется
группа автоморфизмов $L$, сохраняющих $K$:
\[Gal(L/K) = \{\phi: L \xrightarrow{\sim} L \mid \phi|_K = \id \}\]

Нам будут интересны группы Галуа для расширений Галуа.

Обозначим как $\{L:K\}$ количество гомоморфизмов $L \to F$,
где поле $F$ алгебраически замкнуто.

\paragraph*{Замечание} Это определение корректно,
любой такой гомоморфизм пропускается через $\overline{K}$.

\paragraph*{Замечание} $0 < \{L:K\} \le [L : K]$, если $L$ конечно,
равенство достигается для сепарабельных расширений $L \supset K$.

\begin{lemma}
Пусть $L \supset K$ конечно, тогда эквивалентно
\begin{enumerate}
    \item $L \supset K$ --- сепарабельное расширение
    \item $L$ порождается над $K$ сепарабельными элементами
    \item $\{L:K\} = [L : K]$
\end{enumerate}
\end{lemma}
\begin{proof} \indent

$(2) \Rightarrow (3)$ Индукцией по количеству порождающих:
если $\mu_\alpha(x)$ сепарабельный,
то $\{\nicefrac{K[x]}{(\mu_\alpha(x))} : K\} = \deg \mu_\alpha(x)$

$(3) \Rightarrow (1)$ Пусть $\alpha \in L$ несепарабельный.
Тогда $\{K(\alpha) : K\} < [K(\alpha) : K]$ и
\[[L : K] = [L : K(\alpha)] \cdot [K(\alpha) : K] > \{L : K\},\]
что противоречит пункту 3.
\end{proof}

\paragraph*{Следствие} $|Gal(L/K)| \le \{L : K\} \le [L : K]$,
а если $L \supset K$ --- расширение Галуа, то $|Gal(L/K)| = [L : K]$.

\paragraph*{Замечание}
Если $L$ --- поле разложения многочлена $f$ степени $d$,
то $Gal(L/K) \hookrightarrow S_d$. Вложение задаётся действием
гомоморфизмов на корнях $f$.

\subsection*{Основная теорема теории Галуа}

Пусть $L \supset K$ --- расширение Галуа, тогда имеется биекция между
подгруппами в $Gal(L/K)$ и промежуточными
расширениями полей $L \supset E \supset K$:
\[Gal(L/K) \supset H \mapsto L^H = \{a \in L \mid \forall h \in H \: h(a) = a\}\]
\[E \mapsto Gal(L/E)\]

\subsubsection*{Примеры}
\begin{enumerate}
    \item $Gal(\Q(\sqrt{2}) / \Q) = 2\Z$,
          $\phi: \sqrt{2} \mapsto -\sqrt{2}$
    \item Поле разложения $x^3 - 2$ ---
    $\Q(\sqrt[3]{2}, \omega\sqrt[3]{2}, \omega^2\sqrt[3]{2})$,
    где $\omega = \exp(\frac{2\pi i}{3})$.
    Здесь $[\Q(\sqrt[3]{2}, \omega) : \Q] = 3$ и $Gal(\Q(\sqrt[3]{2}, \omega)/\Q) = S_3$.
\end{enumerate}

\begin{lemma}
Пусть $\xi = \exp(\frac{2\pi i}{p})$ --- примитивный корень степени $p$ из единицы.
Его минимальный многочлен --- $x^{p-1}+x^{p-1}+\ldots+1$.

Тогда $Gal(\Q(\xi)/\Q) = \nicefrac{\Z}{(p-1)\Z}
 = \Aut(\nicefrac{\Z}{p\Z})$.
\end{lemma}
\begin{proof}
\[[\Q(\xi):\Q] = \deg(\mu_\xi(x)) = p - 1\]
Каждый $\phi \in Gal(\Q(\xi)/\Q)$ продолжается до
гомоморфизма $\Q(\xi) \to \mathbb{C}$ и полностью определяется тем,
в какой корень $p$-той степени из единицы переходит $\xi$.
\end{proof}

\paragraph*{Пример.} $p = 5$

Степень $[\Q(e^\frac{2\pi i}{5}) : \Q] = 4$, следовательно,
$Gal(\Q(e^\frac{2\pi i}{5})/\Q) = \nicefrac{\Z}{4\Z}$, то есть
существует единственно промежуточное подполе между $\Q$ и $\Q(e^\frac{2\pi i}{5})$.
Пересечение полей --- поле, $\R \cap \Q(e^\frac{2\pi i}{5}) = \Q(\cos(\frac{2\pi}{5}))$.
Отсюда получаем, что $\cos(\frac{2\pi}{5})$ является корнем
многочлена второй степени над $\Q$.

\end{document}
