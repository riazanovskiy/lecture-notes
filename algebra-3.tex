\documentclass[a4paper]{article}
\usepackage{cmap}
\usepackage[T2A,T1]{fontenc}
\usepackage[utf8]{inputenc}
\usepackage[russian]{babel}
\usepackage[left=20mm,right=20mm,top=20mm,bottom=20mm,bindingoffset=0cm]{geometry}
\usepackage[pdfborder={0 0 0}]{hyperref}
\usepackage{indentfirst}
\usepackage{tikz,tikz-cd}
\usepackage{wrapfig}
\usepackage{multicol}
\usepackage{nicefrac}

\usetikzlibrary{matrix, arrows}
\usetikzlibrary{babel}

\makeatletter
\tikzset{
  edge node/.code={%
      \expandafter\def\expandafter\tikz@tonodes\expandafter{\tikz@tonodes #1}}}
\makeatother
\tikzset{
  Subset/.style={
    draw=none,
    every to/.append style={
      edge node={node [sloped, allow upside down, auto=false]{$\subset$}}}
  }
}

\usepackage{amssymb,amsmath,amsthm,amsfonts,amscd}
\usepackage{dsfont}
\usepackage{mathabx}
\usepackage{mathtools}

\newcommand{\R}{\ensuremath{\mathbb{R}}}
\newcommand{\Q}{\ensuremath{\mathbb{Q}}}
\newcommand{\Z}{\ensuremath{\mathbb{Z}}}
\newcommand{\F}{\ensuremath{\mathbb{F}}}

\newcommand{\eps}{\varepsilon}

\let\temp\phi
\let\phi\varphi
\let\varphi\temp

\DeclareMathOperator{\chr}{char}

\DeclareMathOperator{\im}{Im}
\DeclareMathOperator{\Aut}{Aut}
\DeclareMathOperator{\Id}{Id}
\DeclareMathOperator{\ord}{ord}
\DeclareMathOperator{\dsep}{d_{sep}}
\DeclareMathOperator{\di}{d_{i}}

% \renewcommand{\char}{\mathop{\mathrm{char}}\nolimits}

% \usepackage{chngcntr}

\newtheorem{theorem}{Теорема}
\numberwithin{theorem}{section}

\newtheorem{lemma}{Лемма}
\numberwithin{lemma}{section}

\newtheorem*{theorem*}{Теорема}
\newtheorem*{lemma*}{Лемма}

\newtheorem{proposition}{Предложение}
\numberwithin{proposition}{section}

\newtheorem{corollary}{Следствие}
\numberwithin{corollary}{section}

% \let\emph\relax % there's no \RedeclareTextFontCommand
% \DeclareTextFontCommand{\emph}{\bfseries\em}


\begin{document}

\section{Лекция}

\subsection*{Характеристика поля}
Существует единственный гомоморфизм $\phi: \Z \to K$, $\phi(1) = 1_K$.
Возможны два случая:
\begin{itemize}
    \item $\ker \phi = 0$, тогда $\phi$ можно продолжить до гомоморфизма из $\Q$ в $K$,
    $\phi(\frac{n}{m}) = \frac{n_K}{m_K}$,
    тогда $\Q \subseteq K$ и $\chr K = 0$.
    \item
    \begin{minipage}{0.9\linewidth}
        $\ker \phi \ne 0$, тогда $\phi$ пропускается через $\frac{\Z}{\ker \phi}$, где идеал $\ker \phi$ порождается одним элементом~$p > 0$, и $p$ простое.
        Тогда $\F_p \subseteq K$ и $\chr K = p$.
    \end{minipage}
    \begin{minipage}{0.15\linewidth}
    \begin{tikzcd}[row sep=scriptsize,column sep=tiny]
        \Z \arrow[r, "\phi" above] \arrow[d, ->>] & K \\
        \nicefrac{\Z}{\ker \phi} \arrow[ur, hook]
    \end{tikzcd}
    \end{minipage}
\end{itemize}

\begin{theorem}
Если $K$ --- конечное поле, то $|K| = p^n$ для простого числа $p$ и $0 < n \in \Z$.
\end{theorem}
\begin{proof}
$\chr K = p > 0$, следовательно, $\F_p \subseteq K$ и $K$ является векторным пространством над $\F_p$. Если $\dim_{\F_p}K = n$, то $|K| = p^n$.
\end{proof}

\subsection*{Примеры полей}

Пусть $A$ --- коммутативное кольцо с единицей и $I \subset A$ --- идеал.
\begin{proposition}
$\nicefrac{A}{I}$ --- поле тогда и только тогда, когда $I$ --- максимальный.
\end{proposition}
\begin{proof}
Пусть $I$ максимальный, рассмотрим $\phi: A \to \nicefrac{A}{I}$.
Для $x \in \nicefrac{A}{I}$ прообраз $\phi^{-1}((x))$ является идеалом, следовательно
\[\phi^{-1}((x)) \supsetneq I
\Rightarrow \phi^{-1}((x)) = A
\Rightarrow (x) = \nicefrac{A}{I}
\Rightarrow \exists y : \: xy = 1_{\nicefrac{A}{I}}\]

В обратную сторону: пусть $I \subsetneq I'$. $\phi(I')$ --- идеал в
поле $\nicefrac{A}{I}$, следовательно $\phi(I') = \nicefrac{A}{I} \Rightarrow I' = A$.
\end{proof}

\begin{minipage}{0.7\linewidth}
Частный случай: кольцо $A = K[x]$, $f \in K[x]$ --- неприводимый многочлен, тогда $I = (f)$ --- максимальный идеал,
а $\nicefrac{K[x]}{(f)}$ --- поле.
\end{minipage}
\begin{minipage}{0.15\linewidth}
\begin{tikzcd}[row sep=scriptsize,column sep=scriptsize]
    K[x] \arrow[r, "\phi" above] & \nicefrac{K[x]}{(f)} \\
    K \arrow[Subset]{u}{} \arrow[ur, hook]
\end{tikzcd}
\end{minipage}


\textbf{Замечание.}
Существует $\alpha \in \nicefrac{K[x]}{(f)}$, такая что $f(\alpha) = 0$. Если $f(x) = a_d x^d + \ldots + a_1 x + a_0 $ и $\alpha = \phi(x)$,
то $f(\alpha) = f(\phi(x)) = \phi(f(x)) = 0$.

Любой многочлен $g(x) \in K[x]$ можно единственным образом записать как $g(x) = f(x)q(x) + r(x)$, где $\deg r(x) < d$, поэтому
$\{1_K, \phi(x), (\phi(x))^2, \ldots, (\phi(x))^{d-1}\}$ --- базис в
\nicefrac{K[x]}{(f)}.

\subsection*{Расширения полей}
Поле $L$ называется \textbf{расширением} $K$, если $K \subset L$.
Размерность $\dim_K L$ называется \textbf{степенью расширения} --- $[L : K]$.
Если степень $[L : K] < \infty$, то расширение называется \textbf{конечным}.

\begin{proposition} В цепочке расширения $K \subset L \subset F$
\[[F:K] = [F:L] \cdot [L : K]\]
\end{proposition}
\begin{proof}
Пусть $v_1, v_2, \ldots, v_\alpha$ --- базис в $L$ над $K$ и
$w_1, w_2, \ldots, w_\alpha$ --- базис в $F$ над $L$.
Тогда $\{v_i w_j\}$ --- базис в $F$ на $K$.
Действительно, пусть $\alpha \in F$. Тогда
\[\alpha = \sum_j b_j w_j = \sum_j \sum_i a_{ij}v_i w_j \]

Чтобы доказать линейную независимость, рассмотрим
линейную комбинацию $\sum_{i,j} a_{ij}v_i w_j = 0$, где $a_{ij} \in K$.
Переупорядочивая, получим \[\sum_i\left(\sum_j a_{ij} w_j\right)v_i = 0,\]
следовательно \[\forall i \quad \sum_j a_{ij} w_j = 0\]
Отсюда все $a_{ij} = 0$.
\end{proof}

\subsection*{Алгебраические расширения}
Пусть $K \subset L$ --- расширение поле, тогда $\alpha \in L$
называется \textbf{алгебраическим} над $K$, если существует ненулевой
$f \in K[x]$, такой что $f(\alpha) = 0$.

Рассмотрим гомоморфизм $ev_\alpha: K[x] \to L$, $ev_\alpha(f(x)) = f(\alpha)$. Тогда $\alpha$ алгебраический тогда и только тогда,
когда $\ker \alpha \ne 0$.

Если $\alpha$ --- алгебраический над $K$, то $\ker ev_\alpha = (f)$,
где $f$ --- неприводимый и $f(\alpha) = 0$. Такой $f$ называется
\textbf{минимальным многочленом} для $\alpha$. Его степень $\deg f$ обозначается $\deg_K f$.

Для расширения $K \subset L$ и
элементов $\alpha_1, \dots, \alpha_r \in L$
обозначим как $K(\alpha_1, \dots, \alpha_r)$ наименьшее подполе $L$,
содержащее все $\alpha_1, \dots, \alpha_r$.

\begin{proposition}
$\alpha \in L$ алгебраичен над $K$ тогда и только тогда,
когда $[K(\alpha) : K] < \infty$.
\end{proposition}
\begin{proof}

$\Leftarrow$: $K[x] \xrightarrow{ev_\alpha} K(\alpha) \subset L$
$\dim_K K(\alpha) < \infty$, $\dim_K K[x] = \infty$, следовательно
$\ker ev_\alpha \ne 0$ и $\alpha$~---~алгебраический.

$\Rightarrow$: Если $\alpha$ --- алгебраический, то $[K(\alpha) : K] = \deg f$, где $f$ --- минимальный многочлен для $\alpha$.
\end{proof}

Отсюда по индукции получаем, что если $\alpha_1, \dots, \alpha_r$ алгебраические, то $[$K(\alpha_1, \dots, \alpha_r)$ : K] < \infty$.

\begin{theorem}
Пусть $K \subset L$ --- расширение.
Рассмотрим $L_{alg} = \{\alpha \in L \mid  \text{$\alpha$ алгебраично над $K$}\}$.
Тогда $L'$ --- подполе $L$.
\end{theorem}
\begin{proof}
Пусть $\alpha, \beta \in L_{alg}$
\end{proof}


\end{document}
